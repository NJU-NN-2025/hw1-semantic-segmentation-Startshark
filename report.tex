\documentclass[12pt,a4paper]{article}
\usepackage{geometry}
\geometry{a4paper, margin=1in}
\usepackage{fancyhdr}
\usepackage[UTF8]{ctex}
\usepackage{amsmath}
\usepackage{graphicx}
\usepackage{float}
\usepackage{booktabs}
\usepackage{hyperref}
\usepackage{caption}
\usepackage{subcaption}

% 字体设置
\setmainfont{Times New Roman}
\setsansfont{Arial}
\setmonofont{Courier New}
\setCJKmainfont{SimSun}
\setCJKsansfont{SimHei}

\title{\textbf{基于 YOLOv8 的南极动物语义分割实验报告}}
\author{姓名:翟笑晨 \\ 学号:231880394}
\date{\today}

\begin{document}

\maketitle

\section{实验方法}

\subsection{数据获取与标注}
本次实验的数据集来源于NJU大神提供的南极动物图像。为了构建高质量的训练集,我们首先在 Roboflow 平台上对部分图像进行了精细的人工标注,采用\textbf{全监督学习}结合\textbf{迁移学习}的策略进行实验。
\begin{itemize}
    \item \textbf{标注过程}:使用多边形工具(Polygon Tool)精确勾勒出图像中的动物轮廓(如企鹅、海鸥、海豹等)。
    \item \textbf{数据集划分}:在 Roboflow 上将标注好的数据按照一定比例划分为训练集(Train)、验证集(Valid)和测试集(Test)。
    \item \textbf{数据导出}:将处理好的数据集以 COCO JSON 格式导出,文件夹命名为 \texttt{Polar Animal.v9}。
\end{itemize}

\subsection{数据预处理}
为了适配 YOLOv8 模型的输入格式,我们需要将 COCO 格式转换为 YOLO 格式。
\begin{itemize}
    \item \textbf{格式转换}:编写了 \texttt{scripts/convert.py} 脚本,解析 COCO JSON 文件,提取其中的类别信息(Categories)、图像信息(Images)和标注信息(Annotations)。
    \item \textbf{坐标归一化}:将多边形分割点的像素坐标转换为相对于图像宽高的归一化坐标 $(x, y)$,并保存为 YOLO 格式的 \texttt{.txt} 文件。
    \item \textbf{配置文件生成}:脚本自动生成了 \texttt{antarctic.yaml} 配置文件,定义了数据集路径和类别映射关系(如 0: objects, 1: Penguin, 2: Seagull, 3: Seal)。
\end{itemize}

\subsection{模型选择与架构}
本次实验选用了 \textbf{YOLOv8n-seg} 模型。YOLOv8(You Only Look Once version 8)是目前最先进的实时目标检测和分割模型之一。

\subsubsection{YOLOv8 模型原理}
YOLOv8 是一种基于 Anchor-free 的单阶段目标检测与分割模型。其核心思想是将图像划分为 $S \times S$ 的网格,每个网格负责预测中心点落在该网格内的目标。

\begin{itemize}
    \item \textbf{骨干网络}:使用 CSPDarknet53 提取图像特征。该结构引入了 C2f 模块,通过丰富的梯度流信息提升了特征提取能力。
    \item \textbf{颈部}:使用 PANet 结构进行多尺度特征融合,增强了模型对不同大小目标的检测能力。
    \item \textbf{头部}:采用解耦头,分别预测类别、边界框和分割掩码(Mask)。
    \item \textbf{分割分支}:YOLOv8-seg 在检测头的基础上增加了一个 Proto 分支用于生成原型掩码,以及一个 Mask Coefficients 分支用于预测每个实例的掩码系数。最终的实例掩码由原型掩码与系数线性组合并通过 Sigmoid 激活函数得到:
    \begin{equation}
        M = \sigma(\sum_{i=1}^{k} c_i \cdot P_i)
    \end{equation}
    其中 $M$ 是最终掩码,$P_i$ 是原型掩码,$c_i$ 是对应的系数。
\end{itemize}

\subsubsection{损失函数}
YOLOv8 的损失函数由三部分组成:
\begin{equation}
    L_{total} = \lambda_{box} L_{box} + \lambda_{cls} L_{cls} + \lambda_{seg} L_{seg}
\end{equation}
\begin{itemize}
    \item \textbf{分类损失 ($L_{cls}$)}:使用二元交叉熵损失(BCE Loss)计算类别概率的误差。
    \item \textbf{边界框损失 ($L_{box}$)}:结合了 CIoU Loss 和 DFL (Distribution Focal Loss),用于回归边界框的位置。
    \item \textbf{分割损失 ($L_{seg}$)}:计算预测掩码与真实掩码之间的像素级交叉熵损失。
\end{itemize}

\subsection{训练配置}
实验在本地环境进行,配置如下:
\begin{itemize}
    \item \textbf{硬件环境}:NVIDIA GeForce RTX 4060 Laptop GPU (8GB 显存)
    \item \textbf{深度学习框架}:PyTorch, Ultralytics YOLOv8
    \item \textbf{超参数设置}:
    \begin{itemize}
        \item Epochs: 100 (设置了 Early Stopping,实际训练 57 轮)
        \item Batch Size: 8
        \item Image Size: 640
        \item Optimizer: AdamW (lr=0.001)
        \item Model Weights: \texttt{yolov8n-seg.pt} (预训练权重)
    \end{itemize}
\end{itemize}

\section{实验结果}

\subsection{训练过程分析}
模型训练在第 57 轮时触发了早停,图 \ref{fig:results} 详细展示了训练过程中各项指标的变化曲线。我们可以从损失函数和评估指标两个维度来分析模型的训练效果:

\textbf{1. 损失函数 (Losses)}:\textbf{Box Loss}持续下降,表明模型在预测目标位置方面越来越精准;\textbf{Seg Loss}显著降低,说明模型对动物轮廓的分割能力不断增强;\textbf{Cls Loss}的收敛则表明模型能够准确识别目标类别。


\textbf{2. 评估指标 (Metrics)}:\textbf{Precision \& Recall} 在训练初期波动上升后趋于稳定,说明模型在减少误检的同时检出了更多真实目标;\textbf{mAP50} 在前 30 轮增长迅速并稳定在 0.24 左右,表明具备了基础检测能力;\textbf{mAP50-95} 的稳步提升则反映了模型在高质量分割(如边缘贴合)方面的进步。

总体而言,训练集和验证集的损失均呈现收敛趋势,且未出现明显的过拟合现象(验证集损失未反弹),证明模型训练是有效的。

\begin{figure}[H]
    \centering
    \includegraphics[width=1.0\textwidth]{runs/segment/antarctic_yolo_v9/results.png}
    \caption{训练过程中的损失函数与评估指标曲线}
    \label{fig:results}
\end{figure}

\subsection{分割效果展示}

为了直观评估模型的性能,我们使用训练好的最佳模型对测试集进行了推理,并将结果拼接展示。

\begin{figure}[H]
    \centering
    \includegraphics[width=1.0\textwidth]{runs/segment/antarctic_yolo_v9/test_results/stitched_predictions.jpg}
    \caption{测试集上的语义分割预测结果拼接图}
    \label{fig:test_pred}
\end{figure}

如图 \ref{fig:test_pred} 所示,模型能够较好地定位并分割出图像中的企鹅,海鸥等目标。
\begin{itemize}
    \item \textbf{个体识别}:在第一行第四列的图片中,模型准确识别出了岩石前景中的一只企鹅,并给出了较高的置信度(0.94)。
    \item \textbf{群体分割}:在第二行第二列和第四行第四列的图片中,面对冰块上密集的企鹅群,模型虽然能够检测出多个个体,但在重叠区域的分割边界上略显模糊。
    \item \textbf{复杂目标检测}:在第三行第一列的图片中,远处水中的海豹由于目标过小且背景复杂,模型未能成功检出,这反映了模型在小目标检测上的局限性。
\end{itemize}

\section{评价分析}

本次实验使用了多个指标对模型性能进行全面评估,包括精确率(Precision)、召回率(Recall)以及平均精度均值(mAP)。

\subsection{评估指标统计}
表 \ref{tab:metrics} 列出了模型在验证集上的最佳评估结果(Epoch 37)。

\begin{table}[H]
    \centering
    \caption{模型评估指标 (Mask \& Box)}
    \label{tab:metrics}
    \begin{tabular}{lcc}
        \toprule
        \textbf{Metric} & \textbf{Box (检测)} & \textbf{Mask (分割)} \\
        \midrule
        Precision (P) & 0.595 & 0.595 \\
        Recall (R) & 0.228 & 0.228 \\
        mAP@50 & 0.243 & 0.243 \\
        mAP@50-95 & 0.188 & 0.187 \\
        \bottomrule
    \end{tabular}
\end{table}

\begin{itemize}
    \item \textbf{精确率 (Precision, P)}:表示预测为正样本的目标中,真正为正样本的比例。0.595 的值意味着模型预测出的目标中约有 60\% 是正确的,误检率相对较低。
    \item \textbf{召回率 (Recall, R)}:表示所有真实正样本中,被模型正确预测出来的比例。0.228 的值较低,说明模型漏检了较多的目标,这可能与数据集中存在大量模糊、遮挡或小目标有关。
    \item \textbf{mAP@50}:当 IoU (Intersection over Union) 阈值设为 0.5 时的平均精度均值。0.243 的得分表明模型在较为宽松的匹配标准下具有一定的检测能力。
    \item \textbf{mAP@50-95}:在 IoU 阈值从 0.5 到 0.95 步长为 0.05 的范围内计算的平均精度均值。这是一个更严格的指标,0.187 的得分反映了模型在高精度定位和分割边缘贴合度上仍有提升空间。
\end{itemize}

\subsection{混淆矩阵与曲线分析}
图 \ref{fig:perf} 展示了混淆矩阵与 Mask PR 曲线,反映了模型在不同类别上的分类准确度与 Mask 的 P/R 曲线。

\begin{figure}[H]
    \centering
    \begin{subfigure}[b]{0.45\textwidth}
        \includegraphics[width=\textwidth]{runs/segment/antarctic_yolo_v9/confusion_matrix.png}
        \caption{混淆矩阵}
        \label{fig:confusion}
    \end{subfigure}
    \hfill
    \begin{subfigure}[b]{0.45\textwidth}
        \includegraphics[width=\textwidth]{runs/segment/antarctic_yolo_v9/MaskPR_curve.png}
        \caption{Mask PR 曲线}
        \label{fig:curves}
    \end{subfigure}
    \caption{模型性能详细分析图}
    \label{fig:perf}
\end{figure}

\begin{itemize}
    \item \textbf{混淆矩阵分析}:图 \ref{fig:confusion} 中,对角线上的数值代表正确分类的比例。模型在 \textbf{Penguin} 类别上的表现最好,而在 \textbf{Seagull} 和 \textbf{Seal} 上存在较多的误判(被预测为背景或混淆)。这主要是因为这两类样本数量极少,模型未能充分学习其特征。
    \item \textbf{PR 曲线分析}:图 \ref{fig:curves} 展示了不同类别的 Precision-Recall 曲线。曲线下面积越大,模型性能越好。可以看出 Penguin 的曲线包围面积最大(mAP@50=0.377),而 Seagull 和 Seal 的曲线则迅速下降,进一步印证了类别不平衡带来的影响。
\end{itemize}

\section{讨论与总结}

\subsection{模型表现讨论}
\begin{enumerate}
    \item \textbf{优点}:
    \begin{enumerate}
        \item \textbf{实时性强}:YOLOv8n-seg 作为轻量级模型,参数量小,推理速度极快(在 RTX 4060 上仅需约 5ms/张),完全满足实时监测的需求。
        \item \textbf{主要目标识别准确}:对于图像中特征明显、数量最多的企鹅目标,模型展现出了较好的分割效果,能够准确勾勒出其轮廓。
    \end{enumerate}
    \item \textbf{不足与挑战}:
    \begin{enumerate}
        \item \textbf{相似背景干扰}:企鹅的背部颜色与南极岩石极为相近,且训练集中包含岩石背景的负样本较少。这导致模型在复杂背景下容易将岩石误检为企鹅,或者无法将企鹅从岩石背景中清晰分离。
        \item \textbf{密集目标检测}:在企鹅数量较多且密集分布的场景中(如繁殖地),个体之间存在严重遮挡。模型难以逐个区分个体,容易出现漏检或将多个个体分割为一个连通区域(掩码粘连)。
        \item \textbf{小样本与低质量数据}:海豹和海鸥的训练样本数量极少,且部分海豹图片质量较差(如只露出头部、大部分身体在水中、光照条件差)。这导致模型对这两类的特征学习不充分,泛化能力差。
        \item \textbf{RLE 格式解析问题}:在数据预处理阶段,部分复杂的 RLE(Run-Length Encoding)标注无法被脚本正确解析,导致部分训练数据的标签丢失。这直接减少了有效训练样本的数量,限制了模型的性能上限。
    \end{enumerate}
\end{enumerate}

\subsection{总结与改进方向}
本次作业成功实现了基于 YOLOv8 的南极动物语义分割。我们完成了从数据标注(Roboflow)、格式转换、模型微调到推理测试的全流程,并对结果进行了详细分析。尽管模型在主要类别上表现尚可,但在小样本类别和复杂场景下仍有不足。未来的改进方向包括:
\begin{itemize}
    \item \textbf{优化数据转换脚本}:改进 \texttt{convert.py},引入更强大的 RLE 解码库(如 pycocotools),确保所有标注数据都能被正确转换为 YOLO 格式,最大化利用现有数据。
    \item \textbf{数据增强策略}:针对小样本类别(海鸥、海豹),在训练时引入 Mosaic、Mixup、Copy-Paste 等数据增强策略,增加其在训练 batch 中的出现频率和多样性,缓解类别不平衡问题。
    \item \textbf{引入注意力机制}:可以在 YOLOv8 的骨干网络或颈部引入注意力机制(如 CBAM, SE),增强模型对关键特征(如企鹅与岩石的纹理差异)的关注度,提升在复杂背景下的分割精度。
    \item \textbf{模型微调与集成}:尝试使用参数量更大的模型(如最新的 YOLO11),或者采用测试时增强(TTA)和模型集成(Ensemble)技术,以进一步提升模型的泛化能力和鲁棒性。
    \item \textbf{探索弱监督与无监督学习}:考虑到人工标注成本高昂,未来可尝试基于图像级标签的弱监督学习,或利用自监督学习挖掘大量未标注数据的特征,降低对精细标注的依赖。
\end{itemize}

\section*{附录:测试集分割结果}
根据作业要求,我们对指定的测试集数据进行了预测。下图展示了测试集中每张图片的原始图像、Ground Truth 掩码以及模型的预测结果对比。我们将每张图片的原图、掩码、预测结果拼接在一起,以便于直观地评估模型的分割性能。
为了方便直观展示,图片都已经经过后期去除识别框和置信度的标注。

\begin{figure}[H]
    \centering
    \includegraphics[width=0.95\textwidth]{runs/segment/antarctic_yolo_v9/appendix_results/stitched_appendix.jpg}
    \caption{测试集预测结果对比(左:原图,中:掩码,右:预测结果)}
    \label{fig:appendix_stitched}
\end{figure}

\section*{参考文献}
\begin{enumerate}
    \item Jocher, G., Chaurasia, A., \& Qiu, J. (2023). YOLO by Ultralytics (Version 8.0.0) [Computer software]. https://github.com/ultralytics/ultralytics
    \item Lin, T. Y., et al. (2014). Microsoft COCO: Common Objects in Context. In European Conference on Computer Vision.
    \item Wang, C. Y., Bochkovskiy, A., \& Liao, H. Y. M. (2023). YOLOv7: Trainable bag-of-freebies sets new state-of-the-art for real-time object detectors. In Proceedings of the IEEE/CVF Conference on Computer Vision and Pattern Recognition.
\end{enumerate}

\section*{LLMs 协助声明}
部分相关代码(包括 \texttt{scripts/convert.py}, \texttt{scripts/stitch.py} 等)
在编写过程中使用了 GitHub Copilot 进行辅助生成、代码调试。同时该报告也使用了LLM进行文档润色。所有生成内容均经过人工审核与验证。

\end{document}

